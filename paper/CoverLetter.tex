\documentclass[11pt]{letter}
\usepackage[a4paper,left=2.5cm, right=2.5cm, top=1cm, bottom=1cm]{geometry}
\usepackage{hyperref}
\usepackage[osf]{mathpazo}
\signature{Thomas Guillerme}
\address{School of Biosciences\\University of Sheffield\\Sheffield, S10 2TN\\United Kingdom\\guillert@tcd.ie}
\longindentation=0pt
\begin{document}

\begin{letter}{}
\opening{Dear Editors,}

% Simulations are important in biology
Simulating biological data is an important step for testing biological hypothesis.
For example, one might be interested in comparing observed diversification patterns to simulated ones to understand the mechanisms leading to the observed patterns (e.g. Miller et al. 2022 PNAS).
In general, one might be interested in simulated either some phylogenetic hypothesis (topology), some traits or both to generate biologically realistic datasets to be compared to the observed ones.

% There are many tools but they are not modular
Such simulation based approach can be routinely done in many different \texttt{R} packages, such as \texttt{ape} (Paradis \& Schliep 2019 Bioinformatics), \texttt{diversitree} (FitzJohn 2012 MEE), \texttt{TreeSim} (Stadler 2011 Systematic Biology) or \texttt{RPANDA} (Morlon et al. 2016 MEE) to only mention a few.
Each of this packages however was designed of a specific simulation type in mind with a restricted number of options reflecting the objectives of the authors (i.e. simulations where not designed at the core of these excellent packages).
For example, it is possible to simulate traits in \texttt{RPANDA} but not jointly with phylogenies, \texttt{TreeSim} allows for the simualtions of tree topologies and mass extinction events but doesn't allow simulating traits, or \texttt{diversitree} allows to simulate trees and traits but not mass extinction events.

% Here is a treats, a modular package for simulations in biology
Here I propose the \texttt{treats} package: ``TREes And Traits Simulations''.
This package is based on a highly modular architecture allowing users to simulate both trees and traits with the inclusions of events (e.g. mass extinction).
Briefly, it allows and helps users to design their own function (or modify the implemented functions) to design their very own specific simulation scenarios, for example, a birth-death tree with variable speciation and extinction parameters and two 3D dimensional traits (one BM and one OU with multiple beaks) where the speciations becomes trait dependent after reaching a certain number of taxa.
The package comes with an extended documentation and unit tests for each function.
Additionally, this package comes with an in depth Gitbook manual covering most functionalities that will be updated regularly following user requests.

I look forward to hearing from you soon,

\closing{Yours sincerely,}

\end{letter}
\end{document}


%Editors:
% Laura Graham, University of Southampton, UK
% Timothée Poisot, Université de Montréal, Canada
% Samantha Price, Clemson University, USA

%Reviewers
% Emmanuel Paradis emmanuel.paradis@umontpellier.fr
% Will Gearty wgearty@amnh.org
% Tanja Stadler tanja.stadler@bsse.ethz.ch
% Joel Barido joelle.barido-sottani@m4x.org
