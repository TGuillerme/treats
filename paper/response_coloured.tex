% Options for packages loaded elsewhere
\PassOptionsToPackage{unicode}{hyperref}
\PassOptionsToPackage{hyphens}{url}
%
\documentclass[
]{article}
\usepackage{xcolor}
\usepackage{lmodern}
\usepackage{amssymb,amsmath}
\usepackage{ifxetex,ifluatex}
\ifnum 0\ifxetex 1\fi\ifluatex 1\fi=0 % if pdftex
  \usepackage[T1]{fontenc}
  \usepackage[utf8]{inputenc}
  \usepackage{textcomp} % provide euro and other symbols
\else % if luatex or xetex
  \usepackage{unicode-math}
  \defaultfontfeatures{Scale=MatchLowercase}
  \defaultfontfeatures[\rmfamily]{Ligatures=TeX,Scale=1}
\fi
% Use upquote if available, for straight quotes in verbatim environments
\IfFileExists{upquote.sty}{\usepackage{upquote}}{}
\IfFileExists{microtype.sty}{% use microtype if available
  \usepackage[]{microtype}
  \UseMicrotypeSet[protrusion]{basicmath} % disable protrusion for tt fonts
}{}
\makeatletter
\@ifundefined{KOMAClassName}{% if non-KOMA class
  \IfFileExists{parskip.sty}{%
    \usepackage{parskip}
  }{% else
    \setlength{\parindent}{0pt}
    \setlength{\parskip}{6pt plus 2pt minus 1pt}}
}{% if KOMA class
  \KOMAoptions{parskip=half}}
\makeatother
\usepackage{xcolor}
\IfFileExists{xurl.sty}{\usepackage{xurl}}{} % add URL line breaks if available
\IfFileExists{bookmark.sty}{\usepackage{bookmark}}{\usepackage{hyperref}}
\hypersetup{
  pdftitle={Response to reviewers},
  pdfauthor={Thomas Guillerme},
  hidelinks,
  pdfcreator={LaTeX via pandoc}}
\urlstyle{same} % disable monospaced font for URLs
\usepackage[margin=1in]{geometry}
\usepackage{graphicx}
\makeatletter
\def\maxwidth{\ifdim\Gin@nat@width>\linewidth\linewidth\else\Gin@nat@width\fi}
\def\maxheight{\ifdim\Gin@nat@height>\textheight\textheight\else\Gin@nat@height\fi}
\makeatother
% Scale images if necessary, so that they will not overflow the page
% margins by default, and it is still possible to overwrite the defaults
% using explicit options in \includegraphics[width, height, ...]{}
\setkeys{Gin}{width=\maxwidth,height=\maxheight,keepaspectratio}
% Set default figure placement to htbp
\makeatletter
\def\fps@figure{htbp}
\makeatother
\setlength{\emergencystretch}{3em} % prevent overfull lines
\providecommand{\tightlist}{%
  \setlength{\itemsep}{0pt}\setlength{\parskip}{0pt}}
\setcounter{secnumdepth}{-\maxdimen} % remove section numbering
\newlength{\cslhangindent}
\setlength{\cslhangindent}{1.5em}
\newenvironment{cslreferences}%
  {\setlength{\parindent}{0pt}%
  \everypar{\setlength{\hangindent}{\cslhangindent}}\ignorespaces}%
  {\par}

\title{Response to reviewers}
\author{Thomas Guillerme}
\date{2023-11-22}

\begin{document}
\maketitle

Dear Editor,

I am grateful for the associate editor and the two reviewers time and
very encouraging comments that I totally believe have improved the
package, the manual and this manuscript. Please find my response to the
reviewers comments in the following document. I've highlight \textcolor{blue}{the
Associate Editor and both reviewers comments in blue} and my response
black. I'm also sorry for the delay in dealing with the reviewers
comments: I was unemployed until November so didn't work on it until
then.

Best regards,

Thomas

\hypertarget{associate-editor-comments-to-author}{%
\section{Associate Editor Comments to
Author:}\label{associate-editor-comments-to-author}}

\textcolor{blue}{As both reviewers, I really liked the idea of the package, but also
found the manuscript itself a bit hard to follow, which, at least to me,
made it hard to fully appreciate what ``treats'' can currently do and
what are the future additions that might be easily implemented (either
by the developer, or by users). Reading the online manual, I got a
better idea, and hence I think there is still room (metaphorically and I
suspect also physically) for improvement. Sure, the online material has
no space limit constraints, but I do think the manuscript text could be
improved and both reviewers offered some nice suggestions. Apart from
the really good suggestions given by both reviewers I would add the
following suggestions/comments:}

\begin{itemize}
\tightlist
\item
  \textcolor{blue}{1- Are the trait values ``stored'' only at the node and tip (extant
  or extinct), or does it store (e.g.~assuming some small-time
  intervals) the trait values throughout the branch length of each
  lineage? Based on lines 63 to 65 (and 71) I think only at nodes and
  tips, but it might be worth explicitly saying that. Perhaps add the
  word ``only'' in that line. Also, in line 71 is there an extra ``no''
  within the parenthesis?}
\end{itemize}

The trait data in \texttt{treats} is only stored at nodes and tips due
to the code architecture. However, using the argument
\texttt{save.steps} allows to create singleton nodes (nodes with one
ancestor and only one descendant) that can be generated at some
specified or regular time intervals thus effectively saving trait values
as the simulation goes. I've now specified this in the main text (and
removed the typo):

\emph{4. Generating some trait(s) value(s) for the selected lineage
(either a tip or a node - but see the ``Simulating traits section''
below to generate trait values along edges).} line 78-79.

and:

\emph{``Traits are always generated (and stored) only for tips or nodes
and not along edges. However, it is possible so generate trait values at
specific or regular time intervals by using the option
\texttt{save.steps} in the \texttt{treats} function. This will generate
singleton nodes (i.e.~nodes with only one descendant) with associated
trait values.''} line 110-112.

\begin{itemize}
\tightlist
\item
  \textcolor{blue}{2- Regarding the ``storage'' of trait value, if it only stores the
  values at nodes and tips, does it also mean that it only uses trait
  values at those time points to direct the simulation? Does that
  matter? What made me wonder about that was figure 5, and its
  respective intended selective extinction simulation scenario. If I got
  it right, the selective extinction scenario would remove all species
  with positive trait values at the mass extinction event at 15 Mya.
  Looking at the right panel, all species whose direct ancestral node
  had a positive trait value seemed to be removed, but some of those
  with ancestral negative values likely crossed the red line having a
  positive value (even if the ``true trait trajectory'' is not a
  straight as represented, for simplicity I guess, in the figure) and
  were not removed. This might not be always relevant, and I understand
  that some simplifications are necessary, but it might be worth
  explicitly discussing if this is relevant and when. For example, if
  speciation and background extinction rates are small, species are
  likely to have long branches and the trait value it might have at a
  given time point between its ancestral node and its tip value or next
  speciation node might be very different from the value it had at the
  ancestral node (or tip value for that matter). Or is the decision to
  remove the species in the selective mass extinction event one where
  both the ancestral and tip value have to be positive. I guess some
  extra information on how it chooses the species based on the trait
  value (is it the ancestral node value, the tip value, both?) in the
  selective mass extinction regime would also be helpful. I guess the
  issue seems that the trait value is not measured at short time
  intervals (only at the nodes and tips) but a selective extinction
  regime such as this one presented here is a point in time event. }
\end{itemize}

This is an excellent suggestion. I've now reworked \texttt{events} to
now always generates singleton nodes (and associated trait values if
needed) at the time of the event before applying the modification. This
way it will generate extinction based at the trait value of the
extinction time rather than the latest node as it should be the case. It
is then possible to remove the generated singleton nodes to clean up the
tree (to have just data for bifurcating nodes and tips). This is now
updated in the example.

\begin{itemize}
\tightlist
\item
  \textcolor{blue}{3- I have to say that looking at figure 5 it was not trivial to see
  the difference between the two scenarios (I guess that happened
  because most, if not all, species with positive values went extinct in
  the left random panel as well). I know those are realizations of a
  stochastic process and there is nothing wrong here, my concern regards
  simply the visualization. I wonder if choosing another seed might
  result in a set of simulations that are more clearly different, making
  the comparison easier for the eye to pick it up. For example, in the
  online manual, some examples comparing random to selective simulations
  showed a much more disparate pattern.}
\end{itemize}

I've changed the seeds for the plotted random/non-random extinctions
that illustrates the selectivity more. I've also added a horizontal line
at the value 0 to show the positivity of the extinction event for the
non-random one.

\begin{itemize}
\tightlist
\item
  \textcolor{blue}{4- Code box on page 7: the commentary on the second simulation
  scenario reads ``\#Simulate the tree and traits with a random
  extinction event'' and hence is equal to the comment in the first
  simulation scenario in the same box. Perhaps change it to something
  like: ``\#Simulate the tree and traits with a random extinction event
  but based on trait value''}
\end{itemize}

I've changed the comment to
\texttt{\#\#\ Simulate\ the\ tree\ and\ traits\ with\ a\ selective\ extinction\ event}.
code block after line 203.

\begin{itemize}
\tightlist
\item
  \textcolor{blue}{5- I could be wrong but it seems that figure 6 is not cited in the
  text. Please double check and if this is true, please cite this
  figure. }
\end{itemize}

I've added references to figures 5 and 6 in the text on lines 203 and
207.

\begin{itemize}
\tightlist
\item
  \textcolor{blue}{6- All figure captions have some repetition like ``Figure 1: Figure
  1:\ldots.''}
\end{itemize}

I've removed the repetitions.

\begin{itemize}
\tightlist
\item
  \textcolor{blue}{7- Figure 6: I could be wrong, but my intuition was that a selective
  mass extinction event would decrease the disparity more strongly than
  a random extinction mass extinction event. That does not seem to be
  the case. Is it because the ``rule'' for the second extinction
  scenario was that only positive species will go extinct, hence on
  average 50\% of the species (if you start with a value of zero)
  instead of 75\% went extinct? That would make sense, I guess. Or is
  the second extinction scenario one where you also create a 75\% loss
  of species but guarantee that all species with positive values will go
  extinct? Maybe a few words to better explain this would be helpful.
  Perhaps add on lines 109 to 110 something like: ``note that in the
  second scenario, because the ancestral state is zero and we do not set
  the percentage of species to be removed, we expect only 50\% of the
  species to go extinct''. Just an idea.}
\end{itemize}

The results seems a bit different now having chosen a different seed.
However, I have added the Associate Editor's judicious comment to the
main text:

\emph{``Both scenarios illustrate two different types of mass
extinctions but they are not equivalent: because of the ancestral trait
value starting at 0, we expect the second scenario to remove on average
only 50\% of the species (i.e.~half the species are expected to evolve a
trait value above 0). For more details on simulating the effect of mass
extinction and the difficulties to simulate an unambiguous effect of a
mass extinction, see Puttick, Guillerme, and Wills (2020).''} lines
195-199.

\hypertarget{reviewer-1}{%
\section{Reviewer: 1}\label{reviewer-1}}

\textcolor{blue}{This article by Guillerme introduces a new R package treats, which
implements a flexible framework for simulating phylogenies and
associated trait values. Overall, I really like the concept of the
package, and I think it can be very useful for more advanced users who
run many simulations and change their setup often. My comments are
mostly about clarifying the presentation of the different functions. I
also reviewed the manual.}

\hypertarget{comments-on-the-manuscript}{%
\subsection{Comments on the
manuscript}\label{comments-on-the-manuscript}}

\begin{itemize}
\tightlist
\item
  \textcolor{blue}{I had trouble understanding what treats could and could not do from
  the manuscript alone. The introduction discusses several packages
  related to fossil data (FossilSim, paleotree and paleobuddy) which
  gave me the impression that treats could replace them, but it doesn't
  seem that fossilization processes are currently included ? I was also
  a bit confused by the inclusion of time-dependent and
  lineage-dependent simulations in future features, as it seems that the
  current framework should be able to handle these, at least within the
  birth-death process. Overall, having a clearer list of what is
  currently possible or not in treats would be helpful.}
\end{itemize}

I've now clarified what \texttt{treats} does and doesn't in the
introduction:

\emph{``Note that although \texttt{treats} is modular and thus allows to
be used as go to tool for simulating and trees and traits, it lacks the
ready-to-use implemented methods featured in other packages such as
fossilisation and sampling (Barido-Sottani et al. 2019; Stadler 2011;
Rosario Petrucci, Januario, and Quental 2022) or specific
macroevolutionary simulations (Morlon et al. 2016; Puttick, Guillerme,
and Wills 2020).''}. line 63-67

I've also removed the mention of the future time-dependent and
lineage-dependent simulations: these are meant to be internal
implementations allowing the algorithm to track time and lineage IDs
across each edges which will allow even more modularity in the future
but is out of the scope of this paper.

\emph{``For example future planned versions will include abiotic events
and a better integration with the \texttt{dispRity} package.''}. line
218-219

\begin{itemize}
\tightlist
\item
  \textcolor{blue}{I would also have liked some information on the performance of the
  package compared to existing simulation tools. It seems likely that
  the flexibility offered by treats comes at the cost of some
  computational performance, so it would be nice to know how much time
  would be needed to simulate a reasonable dataset (for instance the 50
  replicates used to test the positive extinction vs random
  extinction).}
\end{itemize}

I tried (and saved the tests in \texttt{tests/test-benchmarking.R} for
people to explore) but I'm not convinced how useful it is. First I feel
it's confrontational to the other authors since none of them had the
same objectives in mind. Second I will be biased in making treats look
better (when possible) because I know how it works in much more details.
Third I don't think it's something I want to put forward as a reason of
why to use the package, e.g.~it's slower than TreeSim but faster than
diversitree (under certain parameters) but that's not the reason while
one should use treats over diversitree and TreeSim (the point I am
trying to sell is the modularity, i.e.~if you want to simulate something
specific, treats is good, otherwise, go with what other people have used
before).

\begin{itemize}
\tightlist
\item
  \textcolor{blue}{I didn't understand why the example shown in the manuscript only
  simulated trees up to 30 time units, when the tree used as a basis for
  the comparison is 140 My old (figure 2).}
\end{itemize}

I've now added the following precision:

\emph{``The number of time units in \texttt{treats} is arbitrary and is
not equivalent to millions of years. Using 30 time units here allows to
simulate number of tips in a similar order of magnitude as the ones in
the observed data.''} lines 181-183.

\begin{itemize}
\tightlist
\item
  \textcolor{blue}{I really like the idea of having a library of premade simulation
  templates (l202) and I believe it will be a huge help in making the
  package accessible. Making the examples shown in the manual (section
  8) available as scripts would be a good start for such a library.}
\end{itemize}

I am happy that both reviewers are enthusiastic about the templates
library. Unfortunately I'm not totally sure how to implement it and make
it easier to use and browse by users. So for now I've set up a templated
category of issues on github called
\href{https://github.com/TGuillerme/treats/issues?q=is\%3Aopen+is\%3Aissue+label\%3A\%22simulation+template\%22}{simulation
templates}. I've updated the README file with some indications on how to
save and share such a template.

\hypertarget{minor-comments}{%
\subsubsection{Minor comments:}\label{minor-comments}}

\begin{itemize}
\tightlist
\item
  \textcolor{blue}{ - l71 ``no either'' }
\end{itemize}

Fixed.

\begin{itemize}
\tightlist
\item
  \textcolor{blue}{ - l125 ``where the it''}
\end{itemize}

Fixed.

\begin{itemize}
\tightlist
\item
  \textcolor{blue}{ - l111 the comment in the second example is mismatched with the code
  (random extinction instead of positive extinction)}
\end{itemize}

Fixed.

\begin{itemize}
\tightlist
\item
  \textcolor{blue}{ - l175 the comment in the example is mismatched with the code
  (positive extinction instead of random extinction)}
\end{itemize}

Fixed.

\begin{itemize}
\tightlist
\item
  \textcolor{blue}{ - l178 the text mentions the function trait.extinction but it's not
  used in the code}
\end{itemize}

Fixed.

\begin{itemize}
\tightlist
\item
  \textcolor{blue}{ - l282 missing authors in the reference}
\end{itemize}

This one is just generated as such from the \texttt{mee.csl} reference
style.

\hypertarget{comments-on-the-manual}{%
\subsection{Comments on the manual}\label{comments-on-the-manual}}

\begin{itemize}
\tightlist
\item
  \textcolor{blue}{The main thing the manual is missing in my opinion is a quick
  reference table of the different functions, for people who have
  understood the concepts of the package but need to quickly check what
  format their custom speciation function needs to follow for instance.
  At the moment that information is incomplete (the table in section 4.5
  shows the output types but not what these outputs are) and spread out
  in different places (the description of the function inputs is in
  section 4.2.1 whereas the list of functions is in 4.1).}
\end{itemize}

I've now added a cheat sheet to the github page that shows the key
arguments and their inputs/outputs
\href{https://github.com/TGuillerme/treats/blob/master/inst/gitbook/treats_cheat_sheet.pdf}{here}.
This cheat sheet is rather minimal for now but I'm planning on expanding
on it and making it more interactive (with mouse over menus, etc.) in
the future. Furthermore I've tidied up the summarising of functions in
the manual as suggested by the reviewer.

\begin{itemize}
\tightlist
\item
  \textcolor{blue}{The use of conditions and modifications was quite confusing to me.
  Some of the examples use the modify and condition argument to change
  the birth-death process (section 2.3) whereas some examples simply
  replace the function with a modified version (section 4.4), and I
  could not figure out if these methods were equivalent or if one should
  be preferred to the other. I think the modifications section also
  needs more details about what exactly is being modified, and what the
  expected inputs and outputs of the modification function should be.
  For instance, in the example section 2.3 I don't understand what is
  the ``x'' argument in ``going.extinct'', or why this function returns
  a number instead of a logical. The interface for the conditions in 5.2
  appears to be clearer, but it doesn't match with the example in
  section 2.3 where the ``negative.ancestor'' uses different arguments,
  or with the example in 4.6 where the ``half.the.time'' condition takes
  no arguments. }
\end{itemize}

I have now added clarifications of what the \texttt{"modifiers"} are and
how they should be used. Both in the section 2.3 as a simple explanation
and in the section 4 dedicated to \texttt{"modifiers"}.

\begin{itemize}
\tightlist
\item
  \textcolor{blue}{Finally, I also could not find a full description of the treats
  output object either in the manual or in the package documentation.}
\end{itemize}

I've added a \texttt{@return} section to the \texttt{treats},
\texttt{make.treats}, \texttt{make.bd.params}, \texttt{make.traits},
\texttt{make.modifiers} and \texttt{make.events} documentations as well
as the following section in the manual ``Getting started'' section:

\emph{\texttt{treats} will output either just a tree (class
\texttt{"phylo"}) if no traits were generated or a \texttt{"treats"}
object that contains both a \texttt{\$tree} (\texttt{"phylo"}) and a
\texttt{\$data} (\texttt{"matrix"}) component. Note that this
\texttt{"treats"} class is generalised to most outputs of the package
functions. This allows for a smoother handeling of the objects outputs
such as summarising the content of a \texttt{make.bd.params} output or
visualising a trait output from \texttt{make.traits}.}

\hypertarget{reviewer-2}{%
\section{Reviewer: 2}\label{reviewer-2}}

\textcolor{blue}{Thank you for the opportunity to review this manuscript, titled
``treats: a modular R package for simulating trees and traits''. This
manuscript describes and showcases a new R package that provides users
with an extensive toolbox to jointly simulate phylogenies and evolving
traits. While the example provided within the manuscript gives readers a
small taste of the flexibility of this toolbox, the online manual
(\url{http://tguillerme.github.io/treats.html}) showcases the true
extent of how useful this package will be for evolutionary biologists
far and wide. I really enjoyed reading the manuscript, and I believe it
will be suitable for publication in Methods in Ecology \& Evolution once
my concerns have been addressed.}

\textcolor{blue}{For context, for this review I read the supplied manuscript, tested the
functionality of the R package, and browsed the built-in documentation
and the online supplemental manual. I was unable to review the
functional code of the R package in time for the due date of this
review, but it appears to work properly from my testing. I have one
major concern and a handful of minor concerns. The major concern is that
the overall flow of the manuscript seems a little off. I think this
mostly stems from the worked example preceding the detailed overview of
the package's functions. I think if these two sections were swapped the
manuscript would flow much better. My minor concerns are included at the
end of this review in order of their appearance in the manuscript. I
look forward to reviewing a revised version of this manuscript, if
necessary.}

I have now changed the order of the structure of the manuscript with the
description coming first and the example coming second.

\textcolor{blue}{Best,}

\textcolor{blue}{William Gearty}

\begin{itemize}
\tightlist
\item
  \textcolor{blue}{Lines 27-31: I would enumerate this list.}
\end{itemize}

Done.

\begin{itemize}
\tightlist
\item
  \textcolor{blue}{Line 84: Geological time period names should be capitalized.}
\end{itemize}

Fixed.

\begin{itemize}
\tightlist
\item
  \textcolor{blue}{Figures 2-6: If possible, I think it would be valuable to show what
  code was used to produce these figures in the code snippets. As far as
  I can tell, the plotting is mostly done using functions from the
  package, so this would also be an opportunity to showcase this
  functionality.}
\end{itemize}

Done with the now new inclusion of the \texttt{dispRitreats} code that
streamlines the analysis.

\begin{itemize}
\tightlist
\item
  \textcolor{blue}{Figure 2: The caption should identify what the red line represents
  (the K-Pg boundary?).}
\end{itemize}

Done.

\begin{itemize}
\tightlist
\item
  \textcolor{blue}{Line 111: This may be personal preference, but when I ran this code
  chunk it output a whole bunch of text to the console. It appears to be
  a separate line of text for each replicate in the treats function. I
  believe the intent here is to show the user how much progress has been
  made? Perhaps some sort of progress bar would be more appropriate?}
\end{itemize}

I've now added a \texttt{verbose} option to the \texttt{treats} function
allowing to display the progress or not. Unfortunately the progress bar
suggestion, although comfortable to most users is not feasible
effectively with stochastic data: here the output indicates every re-ran
tree for each replicate. And because the number of runs varies depending
on the random seed, tracking the progress in terms of percentage will
require more computational time (i.e.~estimating how many re-run are
expected for the current seed before increase the progress bar).

\begin{itemize}
\tightlist
\item
  \textcolor{blue}{Lines 106-111: I think you should mention somewhere in this paragraph
  why you are running multiple replicates here (as opposed to the single
  replicate performed earlier in the worked example).}
\end{itemize}

I've added the following sentence at the end of the paragraph:
``\emph{Because of the stochasticity of the simulations, we will repeat
them 50 times (using \texttt{replicates\ =\ 50}) to generate a
distribution of possible simulated scenarios as opposed to a random
single one that could be idiosyncratic.}'' lines 200-203

\begin{itemize}
\tightlist
\item
  \textcolor{blue}{Line 111: I used the same seed as you, but I was unable to reproduce
  Figure 5 (looking at the R markdown file, I see that you use indices
  11 and 10 to make the plots). I think showing the exact code that is
  used here to make Figure 5 would be useful for readers that might be
  ``playing along at home''.}
\end{itemize}

I have streamlined and display the code to allow the users to test the
package while reading the paper.

\begin{itemize}
\tightlist
\item
  \textcolor{blue}{Lines 112-113: You should refer to Figure 6 somewhere in this
  paragraph. It might also be useful to include the code to perform this
  comparison, especially if the functionality is all contained within
  the treats package.}
\end{itemize}

Done. line 207.

\begin{itemize}
\tightlist
\item
  \textcolor{blue}{Line 194: This reference to the manual should have a link to the
  online manual (looks like the link is embedded, but you should also
  have it typed out). I also noticed that the documentation for the
  treats function lists the wrong URL for the online manual.}
\end{itemize}

I've added the URL spelt out.

\begin{itemize}
\tightlist
\item
  \textcolor{blue}{Lines 202-203: Just wanted to note that I really love this idea! I
  think collecting a library of modules will be very useful to users!}
\end{itemize}

As mentioned before, I was not sure how to effectively implement that so
I made a
\href{https://github.com/TGuillerme/treats/issues?q=is\%3Aopen+is\%3Aissue+label\%3A\%22simulation+template\%22}{simulation
templates} issues category on github for now.

\begin{itemize}
\tightlist
\item
  \textcolor{blue}{Line 205: This embedded link does not work. I think it should be
  something like:
  \url{https://github.com/TGuillerme/treats/tree/master/paper}.}
\end{itemize}

Fixed.

\hypertarget{refs}{}
\begin{cslreferences}
\leavevmode\hypertarget{ref-fossilsim}{}%
Barido-Sottani, Joëlle, Walker Pett, Joseph E O'Reilly, and Rachel CM
Warnock. 2019. ``FossilSim: An R Package for Simulating Fossil
Occurrence Data Under Mechanistic Models of Preservation and Recovery.''
\emph{Methods in Ecology and Evolution} 10 (6): 835--40.

\leavevmode\hypertarget{ref-rpanda}{}%
Morlon, Hélène, Eric Lewitus, Fabien L Condamine, Marc Manceau, Julien
Clavel, and Jonathan Drury. 2016. ``RPANDA: An R Package for
Macroevolutionary Analyses on Phylogenetic Trees.'' \emph{Methods in
Ecology and Evolution} 7 (5): 589--97.

\leavevmode\hypertarget{ref-puttick2020complex}{}%
Puttick, Mark N, Thomas Guillerme, and Matthew A Wills. 2020. ``The
Complex Effects of Mass Extinctions on Morphological Disparity.''
\emph{Evolution} 74 (10): 2207--20.

\leavevmode\hypertarget{ref-paleobuddy}{}%
Rosario Petrucci, Bruno do, Matheus Januario, and Tiago Quental. 2022.
``Paleobuddy: An R Package for Flexible Simulations of Diversification
and Fossil Sampling.'' \emph{Methods in Ecology and Evolution} 13 (12):
2692--8.

\leavevmode\hypertarget{ref-treesim}{}%
Stadler, Tanja. 2011. ``Simulating Trees with a Fixed Number of Extant
Species.'' \emph{Systematic Biology} 60 (5): 676--84.
\end{cslreferences}

\end{document}
